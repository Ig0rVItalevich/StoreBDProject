\chapter*{Введение}
\addcontentsline{toc}{chapter}{Введение}
 
Последние годы на рынке товаров народного потребления наблюдается довольно устойчивая тенденция снижения продаж в розничных магазинах, в том числе сетевых, при одновременном росте продаж товаров и услуг через интернет. Интернет-магазины предоставляют возможность покупателям совершать покупки в любое время суток и из любой точки мира. Интернет-магазины парфюмерии становятся все более популярными, так как клиенты имеют возможность получить доступ к широкому ассортименту товаров и выбрать наиболее подходящие ароматы без необходимости посещать физический магазин. 

Существует множество факторов необходимости хорошей базы данных для интернет магазина, вот некоторые из них:

\begin{itemize}
	\item Хранение и организация информации: База данных позволяет эффективно хранить и организовывать информацию о продуктах, клиентах, заказах, платежах и других аспектах работы интернет-магазина. Это обеспечивает систематизацию данных и упрощает доступ к соответствующим данным при необходимости;
	\item Управление инвентарем: База данных позволяет отслеживать наличие товаров, их количество, стоимость и другую информацию, связанную с продуктами, доступными для продажи. Это помогает управлять запасами и предотвращать ситуации, когда товары заканчиваются на складе или есть остатки, которые необходимо реализовать;
	\item Обработка заказов: База данных позволяет хранить информацию о заказах, включая детали заказа, данные клиента, способы оплаты и доставки. Это обеспечивает возможность эффективной обработки заказов, отслеживания их статуса, связи с клиентами и обеспечения актуальной информации о выполнении заказов;
	\item Персонализация и удобство для клиентов: База данных позволяет хранить информацию о предпочтениях клиентов, истории покупок, предыдущих заказах и других персональных данных. Это помогает создать более персонализированный опыт покупателя, предлагая ему рекомендации товаров, специальные предложения или скидки в соответствии с его предпочтениями.
	\item Аналитика и прогнозирование: База данных позволяет проводить анализ данных и выявлять тенденции, например, наиболее популярные товары, поведение клиентов, эффективность маркетинговых кампаний и другие факторы. Эта информация может быть использована для принятия решений по улучшению процессов продажи, разработке маркетинговых стратегий и прогнозирования спроса на товары.
	\item Безопасность данных: База данных позволяет обеспечить безопасность хранения и обработки информации о клиентах, включая персональные данные, платежные данные и другие конфиденциальные сведения. Важно убедиться, что база данных обеспечивает защиту от несанкционированного доступа, взломов и утечек данных, чтобы сохранить доверие клиентов и соблюдать соответствующие правовые и организационные требования.
\end{itemize}

\textbf{Цель данной курсовой работы} --- Разработка базы данных для хранения и анализа информации интернет-магазина парфюмерии.

Для достижения поставленной цели необходимо решить следующие задачи:
\begin{itemize}
	\item формализовать задание, определить необходимый функционал;
	\item провести анализ СУБД; 
	\item описать структуру базы данных;
	\item спроектировать приложение для доступа к БД;
	\item создать и заполнить БД;
	\item реализовать интерфейс для доступа к БД;
	\item разработать программное обеспечение, реализующую поставленную задачу.
\end{itemize}