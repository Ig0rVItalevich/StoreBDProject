\chapter{Исследовательский раздел}
В данном разделе будет проведено исследование реализованной программы.

\section{Технические характеристики}
Тестирование выполнялось на устройстве со следующими техническими характеристиками:
\begin{itemize}
	\item операционная система Ubuntu 20.04.3 LTS [6];
	\item память 7.5 GiB;
	\item процессор Intel(R) Core(TM) i7-8550U CPU @ 1.80GHz × 8 [5].
\end{itemize}

\section{Постановка эксперимента}

\textbf{Цель эксперимента} --- оценка времени генерации модели в зависимости от количества вокселей. В рамках эксперимента моделями являются сферы, построенные на сетках от 5x5x5 до 17x17x17 с шагом 3.

Также необходимо определить зависимость времени регенерации изображения. Для эксперимента рассматривается поворот вокруг оси OY на 10 градусов.

На каждом количестве вокселей эксперимент проводится 10 раз, результатом считается среднее арифметическое серии испытаний. По результатам эксперимента составляются сравнительная табли­ца, а также строятся графики зависимостей.

Во время тестирования устройство было подключено к блоку питания и не нагружено никакими приложениями, кроме встроенных приложений окружения, окружением и системой тестирования.

\section{Результаты эксперимента}

В таблице \ref{time-table} представлены результаты измерения генерации и регенерации моделей при различном количестве вокселей.

\begin{table}[h!]
	\begin{center}
		\caption{Время работы алгоритмов в зависимости от количества вокселей}
		\begin{tabular}{ |c|c|c|c| }
			\hline
			\textbf{\specialcell{Номер \\эксперимента}} & \textbf{\specialcell{Количество \\вокселей}} & \textbf{\specialcell{Время \\генерации(с)}} & \textbf{\specialcell{Время \\регенерации(с)}}\\ \hline
			1 & 68 & 0.11 & 0.06\\ \hline
			2 & 192 & 0.36 & 0.14\\ \hline
			3 & 459 & 0.94 & 0.28\\ \hline
			4 & 868 & 2.88 & 0.51\\ \hline
			5 & 1369 & 7.78 & 0.8\\ \hline
		\end{tabular}
		\label{time-table}
	\end{center}
\end{table}			

На следующем рисунке представлен график зависимости времени работы алгоритма от различного количества вокселей.

\begin{figure}[h!]
	\begin{center}
		\begin{tikzpicture}
			\begin{axis}[
				legend pos = north west,
				xlabel=Количество вокселей,
				ylabel=Время(с),
				minor tick num = 1,
				grid = both,
				major grid style = {lightgray},
				minor grid style = {lightgray!25},
				width = 0.7\textwidth,
				height = 0.5\textwidth]
				
				\addplot[
				green,
				thick,
				mark = o,
				mark size = 3pt,
				semithick,
				] file {assets/generation.dat};
				
				\addplot[
				red,
				thick,
				mark = o,
				mark size = 3pt,
				semithick,
				] file {assets/regeneration.dat};
				
				\legend{
					генерация модели,
					регенерация модели,
				}
				
			\end{axis}
		\end{tikzpicture}
	\end{center}
	\caption{Сравнение скорости работы алгоритмов при различном количестве вокселей.}
\end{figure}

\section*{Вывод}
\addcontentsline{toc}{section}{Вывод}
По результатам эксперимента можно сделать вывод, что время генерации изображения объектов сцены зависит от ко­личества вокселей, составляющих модели, причем зависимость стремится к квадратичной. Время регенерации изображения в разы меньше генерации, ее зависимость от числа вокселей линейна. Разницу во времени можно объяснить работой с внешними файлами и выделением памяти под новую модель.

Таким образом, чем больше вокселей составляют объекты сцены, тем большее требуется процессорное время работы системы, однако чем больше вокселей, тем лучше детализация изоб­ражения.



