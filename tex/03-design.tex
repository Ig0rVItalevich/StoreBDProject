\chapter{Конструкторский раздел}
В данном разделе представлены схемы алгоритмов, выбранных для решения поставленной задачи, описание объектов сцены пользовательские типы и структуры данных, а также приведена структура раализуемого ПО и диаграмма классов.

\section{Общий принцип работы программы}
На рисунках \ref{idef_0}-\ref{idef_1} приведены IDEF0 диаграммы, отражающие организацию работы программы.

\begin{figure}[h!]
	\begin{center}
		\includegraphics[scale=1.05]{assets/idef1.jpg}
	\end{center}
	\caption{IDEF0 диаграмма работы программы}
	\label{idef_0}
\end{figure}

\newpage

\begin{figure}[h!]
	\begin{center}
		\includegraphics[scale=1.05]{assets/idef0.jpg}
	\end{center}
	\caption{Последовательность действий для визуализации модели трованта}
	\label{idef_1}
\end{figure}

\section{Описание объектов сцены}
Сцена состоит из следующих объектов:
\begin{itemize}
    \item \textbf{Сцена} --- плоскость, на которой располагаются трованты, окружающие их объекты и источники света.
    \item \textbf{Тровант} --- трехмерная модель, изменяемая в течение времени в соответствии с заданными пользователем значениями интенсивности роста и его продолжительности, а также точками роста дочерних тровантов. 
    \item \textbf{Объект окружающей среды} --- трехмерная модель объекта сцены, окружающей трованты (дерево, трава, земля).
	\item \textbf{Источник света} --- материальная точка, испускающая лучи света. Положение определяется трехмерными координатами.
\end{itemize}

\section{Типы и структуры данных}

В таблице \ref{types-table} представлены типы и структуры данных, которые необходимо реализовать для разрабатываемого ПО.

\begin{table}[h!]
	\begin{center}
		\caption{Используемые типы и структуры данных}
		\begin{tabular}{ |c|c| }
			\hline
			\textbf{Тип данных} & \textbf{Структура}\\ \hline
			Точка & Координаты X, Y, Z \\ \hline
			Вектор & Начальная и конечная точки \\ \hline
			Полигон & Три точки и значение цвета \\ \hline
			Камера & Точка центра и вектор направленности \\ \hline
			Источник света & Точка центра и интенсивность \\ \hline
			Воксель & \specialcell{Массив полигонов, точка центра, \\длина грани, значение цвета} \\ \hline
			Воксельная модель & Массив вокселей и точка центра \\ \hline
			Октодерево & \specialcell{Точка центра, длина грани начальной \\единицы объема, массив областей объема} \\ \hline
			Сцена & Объекты сцены, источники света, камера \\ \hline
		\end{tabular}
		\label{types-table}
	\end{center}
\end{table}	

\section{Реализуемые алгоритмы}
\textbf{Алгоритм z-буфера} может быть описан последователь­ностью таких шагов:
\begin{enumerate}
	\item заполнить буфер кадра фоновым значением интенсивности или цвета;
	\item заполнить z-буфер минимальным значением $z$;
	\item преобразовать каждый многоугольник в растровую форму в произ­вольном порядке:
	\begin{enumerate}
		\item для каждого пикселя в многоугольнике вычислить его глубину z,
		\item сравнить глубину $z$ со значением $z$ в буфере в этой же позиции. Если $z > z_{buffer}$, то записать атрибут этого пикселя в буфер кадра и заменить $z_{buffer}$ на $z$.
	\end{enumerate}
\end{enumerate}

\textbf{Простая модель освещения} интенсивность рассчитывается по закону Ламберта c учетом рассеянного освещения, представленного константой:
\begin{equation}
I = I_{a}k_{a} + I_{l}k_{d}\cos \theta \qquad 0\leq\theta\leq\pi/2
\end{equation}
$I$ --- интенсивность отраженного света, $I_{i}$ --- интенсивность точечного источника, $k_{d}$ --- коэффициент диффузного отражения ($0\leq k_{d}\leq 1$), $\theta$ --- угол между направлением света и нормалью к поверхности, $I_{a}$ --- интенсивность рассеянного света, $k_{a}$ --- коэффициент диффузного отражения рассеянного света($0\leq k_{a}\leq 1$).

\textbf{Алгоритм рамножения и падения} можно описать в виде последовательности следующих шагов:
\begin{enumerate}
	\item вычисление угла дочернего трованта;
	\item вычисление смещения относительно центра родительского трованта;
	\item рост дочернего трованта;
	\item вычисление массива отпадания при достижении дочерним тровантом достаточного размера;
	\item моделирование отпадания путем замены координат вокселей в соответствие вычисленному массиву отпадания.
\end{enumerate}

\textbf{Алгоритм роста трованта}

Алгоритм, моделирующий рост трованта, использует концепцию октодерева.

\textbf{Октодерево} --- это особый тип дерева разбиения, обычно используемый для объектов в трёхмерном пространстве. Его концепция состоит в том, что параллелепипед, пространственно ограничивающий весь набор данных разбивается на восемь равных частей плоскостями, перпендикулярными каждой из координатных осей. Процесс рекурсивно применяется до тех пор, пока количество элементов данных в каждом вновь образованном октанте не окажется ниже некоторого установленного порога $m>1$ [3].

\newpage

\begin{figure}[h!]
	\begin{center}
		\includegraphics[scale=0.3]{assets/octtree.jpg}
	\end{center}
	\caption{Структура октодерева}
	\label{classes}
\end{figure}

Алгоритм можно описать следующими шагами:
\begin{enumerate}
	\item возврат модели в состояние, при котором ее плоскость параллельна плоскости XZ, путем отмены поворотов;
	\item построение на основе поданной модели октодерева;
	\item в зависимости от области пространства, определяемой октодеревом, вычисляются новые воксели модели;
	\item замена массива вокселей модели на массив, полученный в результате расчета;
	\item возврат модели в исходное состояние.
\end{enumerate}

\newpage

\begin{figure}[h!]
	\begin{center}
		\includegraphics[scale=0.95]{assets/grow.pdf}
	\end{center}
	\caption{Алгоритм роста}
	\label{classes}
\end{figure}

\newpage

На рисунке \ref{classes} представлена диаграмма классов, отражающая разбиение и взаимодействие модулей программы.

\begin{figure}[h!]
	\begin{center}
		\includegraphics[scale=0.3, angle=90]{assets/classes.png}
	\end{center}
	\caption{Диаграмма классов}
	\label{classes}
\end{figure}

\clearpage 

\section*{Вывод}
\addcontentsline{toc}{section}{Вывод}
На основе теоретических данных, полученных из аналитического раздела, были описаны этапы реализации алгоритмов, использующихся в программе. Описаны общий алгоритм принцип и структура программы, типы и структуры данных, а также представлена диаграмма классов. 